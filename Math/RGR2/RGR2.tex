\documentclass[a4paper,12pt]{article}
\usepackage[left=2cm,right=2cm,top=2cm,bottom=2cm]{geometry}
\usepackage[utf8]{inputenc}
\usepackage[english,russian]{babel}
\usepackage{amsmath}
\usepackage{tikz}
\usepackage{pgfplots}
\usepackage{commath}
\usepackage{amsfonts}
\pgfkeys{/pgf/trig format=rad}
\usetikzlibrary{arrows}
\def\letus{%
	\mathord{\setbox0=\hbox{$\exists$}%
		\hbox{\kern 0.125\wd0%
			\vbox to 1.15\ht0{%
				\hrule width 0.75\wd0%
				\vfill%
				\hrule width 0.75\wd0}%
			\vrule height 1.15\ht0%
			\kern 0.125\wd0}%
	}%
}

\begin{document}
\thispagestyle{empty}
\begin{center}
    {\bfseries Национальный исследовательский университет ИТМО}\\
    Факультет программной инженерии и компьютерной техники

    \vspace{20em}

    {\large Расчётно-графическая работа по теме:}\\
    {\Large \textbf{Предел и производная функции одной переменной}}
\end{center}

\vspace{10em}

\begin{flushright}
    Выполнили:\\
    Бавыкин Роман\\
    Баканова Ирина\\
    Лысенко Данила\\
    Остапенко Иван\\
    Группа P3110\\
    Преподаватели:\\
    Беспалов Владимир Владимирович\\
    Вариант 8
\end{flushright}

\vspace{\fill}

\begin{center}
Санкт-Петербург

2020г.
\end{center}
\newpage

\section{Пределы}

Дана последовательность $a_n$ и функция $f(x)$. Исследуйте поведение предложенных величин:

\subsection{Предел последовательности}

$a_n=\sqrt[3]{n^2}\left(\sqrt[3]{\left(3n^2-1\right)^2}-\sqrt[3]{\left(3n^2+1\right)^2}\right)$

\subsubsection{Вычислите предел последовательности при $n\rightarrow\infty$, исследуйте её на монотонность и ограниченность}

$\lim\limits_{n\rightarrow\infty}a_n=
	\lim\limits_{n\rightarrow\infty}\left(\sqrt[3]{n^2}\left(\sqrt[3]{\left(3n^2-1\right)^2}
	-\sqrt[3]{\left(3n^2+1\right)^2}\right)\right)=
	\lim\limits_{n\rightarrow\infty}\left(\sqrt[3]{\left(3n^2-1\right)^2n^2}
	-\sqrt[3]{\left(3n^2+1\right)^2n^2}\right)$\\
$\letus\ a=\left(3n^2-1\right)^2n^2,\ b=\left(3n^2+1\right)^2n^2$\\
$\lim\limits_{n\rightarrow\infty}a_n=
	\lim\limits_{n\rightarrow\infty}\left(\sqrt[3]{a}-\sqrt[3]{b}\right)=
	\lim\limits_{n\rightarrow\infty}\left(\sqrt[3]{a}-\sqrt[3]{b}\right)\cdot
	\frac{\sqrt[3]{a^2}+\sqrt[3]{b^2}+\sqrt[3]{ab}}{\sqrt[3]{a^2}+\sqrt[3]{b^2}+\sqrt[3]{ab}}
	=\lim\limits_{n\rightarrow\infty}\frac{a-b}{\sqrt[3]{a^2}+\sqrt[3]{b^2}+\sqrt[3]{ab}}=$\\
$=\lim\limits_{n\rightarrow\infty}\frac{-12n^4}{\sqrt[3]{81n^{12}-108n^{10}+54n^8-12n^6+n^4}
	+\sqrt[3]{81n^{12}-18n^8+n^4}+\sqrt[3]{81n^{12}+108n^{10}+54n^8+12n^6+n^4}}=$\\
$=\lim\limits_{n\rightarrow\infty}\frac{-12}{\sqrt[3]{81-108n^{-2}+54n^{-4}-12n^{-6}+n^{-8}}
	+\sqrt[3]{81-18n^{-4}+n^{-8}}+\sqrt[3]{81+108n^{-2}+54n^{-4}+12n^{-6}+n^{-8}}}=
	\frac{-12}{3\sqrt[3]{81}}=\frac{-4}{3\sqrt[3]{3}}$\\\\
Последовательность монотонно возрастает.\\
Последовательность ограничена сверху $y = \frac{-4}{3\sqrt[3]{3}}$, и ограничена снизу: $y = \sqrt[3]{4} - \sqrt[3]{16}$


\subsubsection{Постройте график общего члена последовательности в
	 зависимости от номера n.}
\begin{tikzpicture}
	\begin{axis}[
		xlabel={$n$},
		ylabel={$a_n$},  
		%axis lines=middle,
		ymax = -0.922,
		ymin = -0.934,
		xmax=11,
		xmin=0,
		domain = -1:10,
		%xtick=\empty,
		%ytick=\empty,
		scale=2]
		\addplot[dashed, black!40] coordinates{(-2,-0.9254816991341798) (11, -0.9254816991341798)};
		\addplot[dashed, black!40] coordinates{(-2,-0.9234816991341798) (11, -0.9234816991341798)};
		\addplot[dashed, black!40] coordinates{(-2,-0.9245816991341798) (11, -0.9245816991341798)};
		\addplot[dashed, black!40] coordinates{(-2,-0.9243816991341798) (11, -0.9243816991341798)};
		\addplot[dashed, black!40] coordinates{(-2,-0.9244916991341798) (11, -0.9244916991341798)};
		\addplot[dashed, black!40] coordinates{(-2,-0.9244716991341798) (11, -0.9244716991341798)};
		\addplot[only marks, blue,samples=20] table {
			x y
			1 -0.932441047821547
			2 -0.9249585447894372
			3 -0.9245756863557775
			4 -0.9245114264290103
			5 -0.9244938742205362
			6 -0.9244875703969653
			7 -0.9244848682469794
			8 -0.9244835567960762
			9 -0.9244828588587696
			10 -0.9244824600277284
		};
	\end{axis}
\end{tikzpicture}
\subsubsection{Проиллюстрируйте сходимость (расходимость), ограниченность и монотонность последовательности.}
$\varepsilon_1=0,001,\ N_1=2$;\\
$\varepsilon_1=0,0001,\ N_2=3$;\\
$\varepsilon_1=0,00001,\ N_3=6$.\\
\newpage
\subsection{Предел функции}

$f(x)=\left(\frac{5-3x}{1-2x}\right)^{0,3x-3}$

\subsubsection{Вычислите предел функции при $x\rightarrow\infty$ , исследуйте её на монотонность и ограниченность.}

$\lim\limits_{x\rightarrow+\infty}f(x)=
\lim\limits_{x\rightarrow+\infty}
	\left(\frac{5-3x}{1-2x}\right)^{0,3x-3}=
\lim\limits_{x\rightarrow+\infty}\left(
	\left(\frac{5-3x}{1-2x}\right)^{0,3x}\left(
	\frac{1-2x}{5-3x}\right)^3\right)=
\lim\limits_{x\rightarrow+\infty}
	\left(\frac{5-3x}{1-2x}\right)^{0,3x}\cdot
	\lim\limits_{x\rightarrow+\infty}
	\left(\frac{1-2x}{5-3x}\right)^3=
+\infty\cdot\lim\limits_{x\rightarrow+\infty}
	\left(\frac{\frac1x-2}{\frac5x-3}\right)^3=
+\infty\cdot\frac8{27}=+\infty$\\\\
$\lim\limits_{x\rightarrow-\infty}f(x)=
\lim\limits_{x\rightarrow-\infty}
	\left(\frac{5-3x}{1-2x}\right)^{0,3x-3}=
\lim\limits_{x\rightarrow-\infty}\left(
	\left(\frac{5-3x}{1-2x}\right)^{0,3x}\left(
	\frac{1-2x}{5-3x}\right)^3\right)=
\lim\limits_{x\rightarrow-\infty}
	\left(\frac{5-3x}{1-2x}\right)^{0,3x}\cdot
	\lim\limits_{x\rightarrow-\infty}
	\left(\frac{1-2x}{5-3x}\right)^3=
0\cdot\lim\limits_{x\rightarrow-\infty}
	\left(\frac{\frac1x-2}{\frac5x-3}\right)^3=
0\cdot\frac8{27}=0$\\
Функция возрастает, при $x\in(-\infty;-4,0279)\bigcup[6,1694;+\infty)$\\
Функция убывает, при $x\in[-4,0279;0.5)\bigcup[\frac{5}{3};6,1694]$\\
Данная функция ограничена снизу $y=0$

\subsubsection{Постройте график функции в зависимости от x.}

\begin{tikzpicture}
	\begin{axis}[
		xlabel={$x$},
		ylabel={$y$},  
		%axis lines=middle,
		ymax = 10,
		ymin = -2,
		xmax=10,
		xmin=-10,
		domain = -10:10,
		restrict y to domain=-5:20,
		%xtick=\empty,
		%ytick=\empty,
		scale=2]
		\addplot[dashed, black!40] coordinates {(0, 0) (10, 0)};
		\addplot[dashed, black!40] coordinates {(0.5, -2) (0.5, 10)};
		\addplot[very thick, blue,samples=200]{((5-3*x)/(1-2*x))^(0.3*x-3)};
	\end{axis}
\end{tikzpicture}

\newpage
\section{Дифференциал}
Дана задача. Проведите исследование:

Длина телеграфного провода $s=2b\left(1+\frac{2f^2}{3b^2}\right)$, где $2b$ – расстояние между точками подвеса, а $f$ – наибольший прогиб. На сколько увеличится прогиб $f$ , когда провод от нагревания увеличится на $ds$?

\subsection{Составьте математическую модель задачи: введите обозначения, выпишите данные, составьте уравнение (систему уравнений), содержащее неизвестное.}

$ds$ - на сколько увеличился провод;\\
$df$ - на сколько увеличится прогиб;\\
$s'=\frac{ds}{df};\ df=\frac{ds}{s'}=\frac{ds}{\left(2b
	\left(1+\frac{2f^2}{3b^2}\right)\right)'}$
\subsection{Решите задачу аналитически.}

$df=\frac{ds}{\left(2b\left(1+\frac{2f^2}{3b^2}\right)\right)}'=
\frac{ds}{2b\left(\frac{2f^2}{3b^2}\right)'}=
\frac{ds}{\frac{8f}{3b}}=\frac{3b}{8f}ds$

\subsection{Сделайте графическую иллюстрацию к решению задачи. Сверьтесь с аналитическим решением.}

В зависимости от значения b, график функции будет переноситься выше или ниже относительно оси $Oy$ и растягиваться, относительно оси $Oy$. Сделаем иллюстрацию для фиксированного значения $b=1$.

\begin{tikzpicture}
	\begin{axis}[
		xlabel={$f$},
		ylabel={$s$},  
		%axis lines=middle,
		ymax = 10,
		ymin = 0,
		xmax=10,
		xmin=-10,
		domain = -10:10,
		restrict y to domain=-5:20,
		%xtick=\empty,
		%ytick=\empty,
		scale=2]
		
		\addplot[dashed, black!40] coordinates {(1,0) (1,3.5)};
		\addplot[dashed, black!40] coordinates {(2,0) (2,7.5)};
		\addplot[dashed, black!40] coordinates {(-10,3.3) (1,3.3)};
		\addplot[dashed, black!40] coordinates {(-10,7.3) (2,7.3)};
		\addplot[very thick, blue,samples=200]{2*(1+(2*x^2)/(3))};
		
		\node at (axis cs:1.6,0.25) {$df$};
	    \node at (axis cs:-9.5,5.3) {$ds$};
	\end{axis}

\end{tikzpicture}

\subsection{Запишите ответ.}

\section{Наибольшее и наименьшее значение функции}
Дана задача. Проведите исследование:

От канала шириной 2 м под прямым углом отходит канал шириной 4 м. Стенки каналов прямолинейны. Найдите наибольшую длину бревна $l$, которое можно сплавлять по этим каналам из одного в другой.

\subsection{Составьте математическую модель задачи: введите обозначения, выпишите данные, составьте уравнение (систему уравнений), содержащее неизвестное.}

$l=x+y=\frac2{\sin\alpha}+\frac4{\cos\alpha};$

\subsection{Решите задачу аналитически.}
$l'=2(-\frac{\cos\alpha}{\sin^2\alpha}+2\frac{\sin\alpha}{\cos^2\alpha})\ 
l`=0;\ 2\sin^3\alpha-\cos^3\alpha=0;\ 
\tan^3\alpha=\frac{1}{2};\ \tan\alpha=\frac{1}{\sqrt[3]{2}};\\
\cos^2\alpha=\frac{1}{1+\tan^2\alpha}=
\frac{1}{1+\frac{1}{\sqrt[3]{2}}};\ 
\sin^2\alpha=1-\cos^2\alpha=
\frac{\sqrt[3]{4}+\frac{1}{\sqrt[3]{4}}}
	{1+\sqrt[3]{4}+\frac{1}{\sqrt[3]{4}}};\\
l=2\sqrt{\mathstrut{\frac{1+\sqrt[3]{4}+\frac{1}{\sqrt[3]{4}}}
	{\sqrt[3]{4}+\frac{1}{\sqrt[3]{4}}}}}+
	4\sqrt{\mathstrut{1+\frac{1}{\sqrt[3]{2}}}}$

\subsection{Сделайте графическую иллюстрацию к решению задачи. Сверьтесь с аналитическим решением.}

\begin{tikzpicture}
	\draw (0,8) -- (0,0) -- (8,0);
	\draw (2,8) -- (2,4) -- (8,4);
	\draw (0,6) -- (6,0);
	\node at (0.3,5.3) {$\alpha$};
	\node at (1,8) {2м};
	\node at (8,2) {4м};
	\node at (1,5.5) {$x$};
	\node at (4,2.5) {$y$};
\end{tikzpicture}

\subsection{Запишите ответ.}

$l=2\sqrt{\mathstrut{\frac{1+\sqrt[3]{4}+\frac{1}{\sqrt[3]{4}}}
	{\sqrt[3]{4}+\frac{1}{\sqrt[3]{4}}}}}+
	4\sqrt{\mathstrut{1+\frac{1}{\sqrt[3]{2}}}}$


\section{Исследование функции}
Даны функции $f(x)$ и $g(x)$. Проведите поочерёдно их полные исследования:

$f(x)=\frac{2x^3-5x^2+14x-6}{4x^2};$

$g(x)=\frac{1}{2}e^{\sqrt{2}\cos x}$

\subsection{$f(x)$}

\subsubsection{Найдите область определения функции.}

$D=\mathbb{R}\backslash\{0\}$

\subsubsection{Проверьте, является ли функция чётной (нечётной), а также периодической, и укажите, как эти свойства влияют на вид графика функции.}

$f(-x)=\frac{2(-x)^3-5(-x)^2+14(-x)-6}{4(-x)^2}=
\frac{-2x^3-5x^2-14x-6}{4x^2};\\
f(-x)\neq f(x);\ f(-x)\neq -f(x)\Rightarrow$ функция не является ни чётной, ни нечётной.\\
Функция не является периодичной.

\subsubsection{Исследуйте функцию на нулевые значения и найдите промежутки ее знакопостоянства.}

$f(x)=0,\ x = \frac12\\
f(x)>0,\ \mbox{при}\ x\in(\frac12;+\infty);\\
f(x)<0,\ \mbox{при}\ x\in(-\infty;\frac12)\backslash\{0\}$


\subsubsection{Исследуйте функцию с помощью первой производной: найдите интервалы монотонности и экстремумы функции.}

$f'(x)=\frac12-\frac7{2x^2}+\frac{3}{x^3};$\\
$f'(x)=0;\ \frac12-\frac7{2x^2}+\frac{3}{x^3}=0;\\
x_{min}=\{-2; 3\};\ x_{max}=-1.$\\
Функция убывает, при x
	$\in(-\infty;-2]\bigcup[-1;3]\backslash\{0\}$;\\
функция возрастает, при $\in[-2;-1]\bigcup[3;+\infty)$

\subsubsection{Исследуйте функцию с помощью второй производной: найдите интервалы выпуклости (вогнутости) и точки перегиба функции.}

$f''(x)=\frac7{x^3}-\frac9{x^4};$\\
Точка перегиба - $x=\frac97,$\\
функция выпукла, $x\in(-\infty;\frac97)\backslash\{0\},$\\
функция вогнута, $x\in(\frac97;+\infty)$

\subsubsection{Проверьте наличие вертикальных, горизонтальных и наклонных асимптот графика функции.}

$x=0$ - точка разрыва второго рода, является вертикальной асимптотой;\\
уравнение наклонной асимптоты: $y=\frac12x-\frac54$

\subsubsection{Найдите точки пересечения графика с координатными осями и (при необходимости) найдите значения функции в некоторых дополнительных точках.}

График пересекает ось $Ox$ в точке $x=\frac12$

\subsubsection{Постройте график. Отметьте на нём все результаты исследования.}

\begin{tikzpicture}
	\begin{axis}[
		xlabel={$x$},
		ylabel={$y$},  
		%axis lines=middle,
		ymax = 10,
		ymin = -10,
		xmax=10,
		xmin=-10,
		domain = -10:10,
		restrict y to domain=-15:15,
		%xtick=\empty,
		%ytick=\empty,
		scale=2]
		\addplot[dashed, black!40]{0.5*x-1.25};
		\addplot[very thick, blue,samples=200]{(2*x^3-5*x^2+14*x-6)/(4*x^2)};
	\end{axis}
\end{tikzpicture}

\subsection{$g(x)$}

\subsubsection{Найдите область определения функции.}

$D=\mathbb{R}$

\subsubsection{Проверьте, является ли функция чётной (нечётной), а также периодической, и укажите, как эти свойства влияют на вид графика функции.}

Функция является чётной:\\
$g(-x)=\frac12e^{\sqrt2\cos(-x)}=\frac12e^{\sqrt2\cos x}=g(x);$\\
Функция является периодической, период $2\pi$:\\
$g(x+2\pi)=\frac12e^{\sqrt2\cos(x+2\pi)}=
\frac12e^{\sqrt2\cos x}=g(x).$

\subsubsection{Исследуйте функцию на нулевые значения и найдите промежутки ее знакопостоянства.}

Функция не имеет нулевых значений, функция возрастает, при $x\in\mathbb{R}$.

\subsubsection{Исследуйте функцию с помощью первой производной: найдите интервалы монотонности и экстремумы функции.}

$g'(x)=-\frac{\sqrt2}{2}\sin x\cdot e^{\sqrt2\cos x},$\\
$x_{max}=2\pi n,\ x_{min}=(2n+1)\pi,\ n\in\mathbb{Z},$\\
функция возрастает, при $(2n+1)\pi\leq x\leq2\pi n,\ 
	n\in\mathbb{Z},$\\
функция убывает, при $2\pi n\leq x\leq(2n+1)\pi,\ n\in\mathbb{Z}.$

\subsubsection{Исследуйте функцию с помощью второй производной: найдите интервалы выпуклости (вогнутости) и точки перегиба функции.}

$g''(x)\frac{\sqrt2}{2}e^{\sqrt2\cos x}
	\left(\sin^2x-\cos x\right),$\\
точки перегиба: $x=\pm\arccos\frac{-1+\sqrt5}2+2\pi n,\ 
	n\in\mathbb{Z},$\\
функция выпукла, при $x\in(-\arccos\frac{-1+\sqrt5}2+2\pi n;
	+\arccos\frac{-1+\sqrt5}2+2\pi n),\ n\in\mathbb{Z},$\\
функция вогнута, при $x\in(+\arccos\frac{-1+\sqrt5}2+2\pi n;
-\arccos\frac{-1+\sqrt5}2+2\pi (n+1)),\ n\in\mathbb{Z}.$
\subsubsection{Проверьте наличие вертикальных, горизонтальных и наклонных асимптот графика функции.}

У данной функции асимптоты отсутствуют.

\subsubsection{Найдите точки пересечения графика с координатными осями и (при необходимости) найдите значения функции в некоторых дополнительных точках.}

График пересекает ось $Oy$, при $y=\frac12e^{\sqrt2}$

\subsubsection{Постройте график. Отметьте на нём все результаты исследования.}

\begin{tikzpicture}
	\begin{axis}[
		xlabel={$x$},
		ylabel={$y$},  
		%axis lines=middle,
		ymax = 10,
		ymin = -1,
		xmax=3*pi,
		xmin=-3*pi,
		domain = -3*pi:3*pi,
		%restrict y to domain=-1:3,
		%xtick=\empty,
		%ytick=\empty,
		scale=2]
		\addplot[very thick, blue,samples=200]{0.5*e^(sqrt(2)*cos(x))};
	\end{axis}
\end{tikzpicture}

\end{document}